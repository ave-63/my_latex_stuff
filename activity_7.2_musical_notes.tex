\documentclass[letterpaper, 12pt]{article}
\usepackage{boneside}
\usepackage[document]{ragged2e}

\begin{document}
\begin{center}   \textbf{Class Activity - 7.2 - Musical Notes and Exponential Functions} 
\end{center} 

Sound is actually air vibrating slightly between high and low pressure, very quickly. The \emph{frequency} of a musical note is a measure of the speed at which the air is vibrating, and it's measured in \emph{Hertz}, aka \emph{Hz}, the number of vibration cycles (from high pressure to low pressure and back to high) in the air every second. For example, the note $A_4$ has frequency $440$ Hz, so the air vibrates 440 times each second when an $A_4$ is played. Every time you go up an \emph{octave}, such as from $A_4$ up to $A_5$, the frequency doubles, so $A_5$ is $880$ Hz. There are 12 notes in any octave that are each a \emph{half-tone} apart; for example, $A$, $A\sharp$, $B$, $C$, $C\sharp$, $D$, $E$, $E\sharp$, $F$, $F\sharp$, $G$, $G\sharp$.  To make notation easier, we'll let $x$ be the number of half-tones that a note is above $A_4$, and let $F(x)$ be the frequency of that note. For example, we'll use $x=12$ for $A_5$, so that $F(12)=880$.

\textbf{1.} Fill in the table with the frequencies of several $A$ notes. Try to find a formula for $F(x)$ (\emph{this is very tricky and will take some trial and error. Check if your formula works by plugging in $x=12$ and $x=24$}).

\begin{tabular}{|c | c | c | c | c | c |} \hline
	$x$ &	0 & 12 & 24 & 36 \\ \hline
	$F(x)$ & \hspace{8ex} &\hspace{8ex}  & \hspace{8ex} & \hspace{8ex} \\ \hline
\end{tabular}

\vspace{3in}
\textbf{2.} By what percentage does the frequency increase, every time $x$ increases by 1?

\vfill
\textbf{3.} Now use your formula to find all the notes in one octave of the key of A major:

\begin{tabular}{|c | c | c | c | c | c |c |c |} \hline
	$x$ &	0 ($A$)  & 2 ($B$) & 4 ($C\sharp$) & 5 ($D$) & 7 ($E$) & 9 ($F\sharp$) & 11 ($G\sharp$)   \\ \hline
	$F(x)$ & \hspace{8ex} &\hspace{8ex}  & \hspace{8ex} & \hspace{8ex} & \hspace{8ex} & \hspace{8ex} & \hspace{8ex} \\ \hline
\end{tabular}

\vfill
\textbf{4.} One way two notes sound harmonious is if the ratio of their frequencies is approximately 1.5. Find two notes whose ratio is about 1.5.

\vfill
\textbf{5.} A population of bacteria starts with 100 bacteria on day zero, and increases by 60\% every 7 days. Find a formula for $P(t)$, the number of bacteria after $t$ days. (hint: how is this similar to the situation with problem \textbf{1}?)

\vfill
\end{document}