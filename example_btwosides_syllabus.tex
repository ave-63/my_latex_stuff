\documentclass[12pt, twoside]{article}
\usepackage{btwosides}


\begin{document}
\begin{center}
	\large{\textbf{Intermediate Algebra - Math 125 - Section 20575}}
			
	\large{\textbf{MoTuWeTh 9:35 - 10:45 in 8210}}
\end{center}

\subsection*{Instructor: Ben Smith}
	
   \noindent \textbf{Office Hours in 8346A:} TuTh 11:00 - 1:00, and 2:00 - 3:00. Also available MW 2:15 - 3:30 by appointment.
		
	\noindent \textbf{Email:} smithbt@piercecollege.edu

\subsection*{Required Materials}
      \textbf{Textbook:}  \emph{Intermediate Algebra - A Modeling Approach}, by Katherine Yoshiwara, published by xyztextbooks, ISBN 978-1-936368-87-7. It costs about \$60.00 in the student store.
      
      \noindent\textbf{Activities Workbook:} \emph{Intermediate Algebra - A Modeling Approach Activities Workbook} by Katherine Yoshiwara, published by xyztextbooks, ISBN 978-1-936368-35-8. It costs about \$22.00 in the student store.
      
      \noindent\textbf{Calculator:} You are required to have a TI-83 or TI-84 graphing calculator. You'll learn how to do lots of things with it in the lecture videos, and there will be many types of problems you won't be able to do without one. I recommend buying a used one on amazon or ebay for about \$40 or \$50.
   
      \noindent\textbf{HW Binder:} You'll need a 3-ring binder to keep your work, including lecture notes, HW, and tests.  Each day, you'll turn in a few pages from this at the beginning and end of class. You'll need the binder so it's easy to put things back in the right place.
   
      \noindent\textbf{Course Canvas Site:} Go to https://ilearn.laccd.edu. Your login credentials are the same as the new PeopleSoft Student Information System and your laccd.edu email. Here you'll find discussion boards, the list of HW items, the class google calendar, helpful links, etc. You're required to be aware of upcoming dates on the class calendar so I recommend subscribing it to whatever calendar software you use on your phone or PC.  

\subsection*{Grades}

   \begin{center}
      \begin{tabular}{l | l}
         60\% - Five Midterm Exams & $85\% \le A$ \\
         25\% - Final Exam         & $75\% \le B < 85\%$ \\
         15\% - HW                 & $65\% \le C < 75\%$ \\
                                   & $55\% \le D < 65\%$ \\
                                   & \hspace{3.5em}$F < 55\%$
      \end{tabular}
   \end{center}
   
   \noindent   \textbf{Midterm Exams:} You are responsible for knowing midterm dates which are posted on the class google calendar. I will not allow make-ups of midterms without documentation of a serious emergency. Things that are \emph{not} serious emergencies: doctor's appointments, car trouble, traffic, etc. The lowest midterm score will be dropped and replaced by your final exam score, if that improves your grade.
      
   \noindent \textbf{Final Exam:} The final will be the MET, the same final exam that all Pierce Math 125 students take. It is Saturday Dec 8, 1:00 - 3:15  in a classroom to be announced. It will cover the whole course.

\newpage

\subsection*{Flipped Classroom and HW}
   \textbf{What is a flipped classroom?} Instead of watching lectures in class, you'll watch youtube video lectures at home. Class time will mostly be spent doing practice HW problems, group activities, and getting help. There will occasionally be a mini-lecture in class when lots of people need help on the same topic.
   
   \noindent\textbf{HW Format:}  The HW for the course is divided into \emph{pre} and \emph{post} HW assignments for each section. Each assignment is made of a few \emph{items}, listed on the \texttt{hw\_items\_ch\_x.pdf} files on canvas. Pre-HW is usually one of two types: \textbf{read the textbook}, or \textbf{watch a video}. Post-HW is usually to \textbf{do some problems} from the problem set at the end of the section. You'll keep all of these different completed items in your HW binder. Make sure your name is on each page, with the assignment clearly labeled.
   
   \begin{itemize}
      \item \textbf{Read the textbook:} On your paper, you should take notes, as if you were watching a lecture. You should have all examples worked out, all new vocabulary terms defined, and any important formulas or facts written down. Make sure to write down questions about anything you don't understand. I may also ask you to do a couple problems on your own, and write your work and answers for those on your paper too. Reading a math textbook is \emph{slow}, even for mathematicians. You may have to read a sentence a few times before you understand it. If you try to read it at the same speed you read Harry Potter, you're guaranteed to get lost.
      \item \textbf{Watch a video:} This involves watching a youtube video. On your paper, take notes. I intentionally make the videos pretty fast, so most students have to frequently pause it to catch up with notes, or rewind to hear something repeated. You should work out each example as much as you can, with the video paused, before watching the solution worked out. When you don't understand something, write down your question, and include the play time (location in the video) so that you can ask for help in class with the video on your phone. Usually, I'll ask you a couple questions from within the video to do on your paper.
      \item \textbf{Do some problems:} This is usually a list of regular HW problems from the book. Show all work and box answers on your paper. 
   \end{itemize}
   
   \noindent\textbf{What is due when?} On the course google calendar, it says which assignments are due each day. Usually this includes Pre-HW for the section we're covering that day, and Post-HW from the section covered a week ago.

\subsection*{Attendance and Drop Policy}   
   You are required to be ready to work at the beginning of class every class session, and stay until the end. If you arrive late or leave early, you won't be able to turn in the HW items that were due at that time. If you miss 5 classes, or are excessively repeatedly tardy, you may be excluded from the class. Plan on arriving 5 minutes early, or 20 minutes early, or whatever it takes to make sure that you are on time even if there’s traffic or parking problems that might slow you down. If you have 3 or more absences (or times you’ve been late), consider yourself “on probation,” and talk to me to make a plan to make sure you don’t get dropped.

\newpage
\begin{center}
   \large{\textbf{Syllabus Addendum 1 - Advice, Behavior, Course Content}}
\end{center}

\subsection*{How to Succeed in this Course}
\begin{itemize}
   \item Math is something you learn by doing. Simply watching videos will not lead to understanding. Watch the videos \emph{actively}, with your pencil and paper out, pausing frequently, trying to answer the problem on your own ahead of the video. Figuring out \textbf{one} problem on your own is better than passively watching \textbf{five} problems, simply taking notes as you go.
   
   \item It is better to work a bit every day (1 or 2 hours) than once a week for a large chunk of time (5 or 10 hours).
   
   \item Work with others outside of class if possible. It's extremely helpful to talk about math with other people. You can ask them for help explaining something in a video if you watch it together, or give each other hints on HW problems.
   
   \item Don't be afraid of making mistakes on HW. Everyone makes mistakes, but some people are defeated by mistakes, and some people use them to grow. If you make a mistake, and check your answer, and then find and correct your mistake, then you've learned much more than someone who just got it right the first time.
   
   \item Don't be afraid of getting stuck on HW. Just like with making mistakes, getting stuck is a great opportunity to improve if you react the right way. Here are some things to do when you get stuck:
   \begin{itemize}
      \item Try writing down EVERYTHING you know about the problem, including any relevant equations or facts from your notes or the book, and any ideas you have. Sometimes you might inspire yourself, and anyway, it will help to have it all laid out as you work on the problem. Don't worry about using lots of paper, which is a cheap, renewable resource.
      \item See if you can find a similar example in your notes or the book. 
      \item If there are two ways you could go with a problem, and you're not sure which way to go, try both ways, next to each other on your paper. Then you can check your answer in the back and figure out which one is probably right.
      \item Try writing down a specific question that you could ask Mr Smith, a classmate, or a tutor in the CAS. Sometimes simply writing down your thoughts helps.
      \item Don't spend more than 5 or 10 minutes stuck in the same spot. Mark that you were stuck on your paper, get help later if possible, and move on to something else.
   \end{itemize}    
   
   \item If you have a disability which may require classroom or test accommodations, please see me as soon as possible during office hours or after class. If you have not already done so, please register with the office of Special Services. Their office is in the new Student Services Building, 4800.
\end{itemize}


\newpage

\subsection*{How to Help Others Succeed in this Course}
\begin{itemize}
   \item I encourage you to work together, but stop conversations when I'm talking to the whole class.
   \item No phone games or videos allowed while class is in session. Avoid texting and social media. 
   \item Do not be ashamed of your ignorance, and do not shame others for theirs. You are all in math 125 and in no position to judge. They are not that far behind you and might catch you up this semester if they work hard. 
   \item Remember we all have very different backgrounds. If you are gifted in one way or another, don't flaunt it. If you got an A, don't say ``that test was easy'' to someone who worked hard to get a C.
   \item One of the best ways to get better at math is to try to explain what you know to someone else. If someone asks you a question, they are doing you a favor. You might not be used to talking about math, not know the right words, and struggle to explain yourself. This is OK. Take your time and don't feel bad if the person you're helping doesn't understand everything.
\end{itemize}

\subsection*{Topics Covered}

We'll be covering almost the entire textbook. There are two sections that might be skipped if we are running short on time. To see details of course content, search for the math 125 Course Outline of Record at ecd.laccd.edu.

\subsection*{Student Learning Outcome}
   Upon successful completion of the course the student will be able to perform a real-world task requiring Intermediate Algebra mathematics that demonstrates meaningful application of essential knowledge and skills.  Examples of essential knowledge and skills at the Intermediate Algebra level may include but are not limited to: a) Representing and analyzing basic functions and their applications using tables, graphs, and equations, b) Using and interpreting function notation in both algebraic and graphical contexts, c) Writing and analyzing linear models for functions with constant rate of change, d) Graphing linear equations and interpreting slope as a rate of change in real world situations, e) Modeling problems involving two or more unknowns by writing and solving systems of equations or inequalities, f) Formulating and analyzing quadratic models, such as projectile motion, revenue functions, problems involving area or the Pythagorean theorem, and applications of conic sections, such as planetary orbits, g) Applying and interpreting exponential models such as population growth and compound interest, and logarithmic scales such as pH and earthquake magnitude, h) Using exponents and radicals to analyze power function models in applications such as direct and inverse variation and allometry (scaling in Physiology).


%\subsection*{How to Use Office Hours}
%First of all, you should come to office hours to get math help at least once this semester. It is a helpful resource for you, and I really enjoy one-on-one tutoring. I have one rule to keep it that way:
%\begin{enumerate}
 %  \item You must bring specific questions about math. You can't use office hours to make up for missed class, or get your own private mini-lecture because you don't understand a topic.
%\end{enumerate}

%\noindent Here are some examples of \emph{good} questions to bring to office hours:
%\begin{itemize}
%   \item ``I don't know how to get started on number 15 here.''
%   \item ``I didn't understand what you were talking about in the video from HW Item 73, between 1:30 and 1:50.''
%   \item ``I am a week behind on HW because I got the flu. I don't think I'll have time to catch up on all of the material. Can you tell me which are the most crucial HW items for me to focus on first, so that I can understand enough to continue in the class?''
%   \item ``I'm considering dropping the class because I don't think I've been getting D's on tests and I don't know if I'll be able to catch up. What do you think my chances are?''
%\end{itemize}

%\noindent Here are some examples of \emph{bad} questions to bring to office hours:
%\begin{itemize}
%   \item ``I didn't understand anything in this video.''
%   \item ``I don't understand anything about logarithms. Can you help me?'' No, I cannot teach you everything about logarithms in office hours.
%\end{itemize}

\newpage
\begin{large}
\begin{center}
   \textbf{Syllabus Addendum 2 - Academic Honesty}
\end{center}
\end{large}

\subsection*{Copying and Getting Help on HW}

You'll get the most out of a HW problem if you do everything yourself, and struggle through it. But of course, some times you're going to get stuck, and you're going to get help in one form or another, and that's ok too. You might ask your friend, and they'll explain to you how they did a problem. You might check your answer in the back of the book and find you made a mistake and fix your work. You might just be stuck, and write down the answer from the back, just so you have it for reference later. You are allowed to do all of these things. But you must follow the \emph{one rule:}

\begin{enumerate}
   \item Whenever you get help from a person, or copy an answer, or use any resources other than your own brain, pencil, and calculator, you must write a little note giving credit.
\end{enumerate}

I will maybe take away points depending on how much help you needed. For example, if you turn in a HW Item where you copied every single answer from the back, you'll probably get zero or maybe half credit depending on how much you tried on your own. If you turn in a HW Item where you got help from a classmate for two problems (but didn't copy), you'll probably get full credit. If you turn in HW where you copied answers but didn't give credit, you'll get a bunch of negative points.

Here are some examples of how to give credit in your HW:

\noindent\includegraphics[width=.4\linewidth]{PDFAnnotator_2018-01-17_10-46-54.png}
\includegraphics[width=.4\linewidth]{PDFAnnotator_2018-01-17_11-03-00.png}

\includegraphics[width=.4\linewidth]{PDFAnnotator_2018-01-17_10-53-10.png}

\newpage 
\subsection*{Cheating on Tests Policy}
I take cheating on tests seriously. It is important to me, and important to students who don't cheat, that students can't cheat and get away with it. I hope that having clear rules during tests can convince everyone not to consider cheating in the first place. First of all, here are the rules during tests:

\begin{enumerate}
   \item Do not communicate with other students taking the test in any way. Don't ask them for an eraser or share a calculator with them. Raise your hands and ask me for these things if you need them. Don't give your friend a thumbs up when you recognize a problem you luckily studied together. 
   \item Do not look in the general direction of others' tests. I probably can't tell the difference between you checking out someone's tattoo, and you copying their answers.
   \item Go to the bathroom before a test starts so you don't have to go during the test.
   \item Phones and smartwatches must be out of sight at all times. The only exception is that if your phone is ringing, you can get it out to turn it off.
   \item There are more details under ``Academic Honesty'' in the course catalog.
\end{enumerate}

{\centering\textbf{What Happens if the Rules are Broken:} \par}

\begin{itemize}
   \item The first time in the semester that I see you attempting to communicate or peek, you'll get a warning. The second time I see you attempting to communicate or peek, I will give you a zero on that test and submit a report to the responsible dean. If you have been warned once, and you're afraid your eyes might wander by mistake, I recommend sitting in the front, where it's harder to accidentally look towards other students' papers.
   \item If I see your phone or smartwatch on your lap, on your chair, on your wrist, or anywhere else within your sight, you'll get a zero on the test and I'll submit a report.
   \item If you actually succeed in copying, and write down someone else's answers, and I have proof of it, you'll get a zero on the test and I'll submit a report. Usually the proof is in the form of the same mistake made on both papers, in such a way that is highly unlikely to be a coincidence. 
\end{itemize}

\newpage




   
%\subsection*{Fundamental Skills}

%Some math problems are more important than others. Think of the content of this course as a tree. Some kinds of problems are like the small branches or leaves of a tree. Some kinds of problems are like the trunk of the tree. If your ``tree of knowledge'' is missing a some leaves, you probably won't get an A but you'll probably pass. But if there's a big hole in the trunk, the tree might die. Students have a hard time telling the difference between Trunk problems and Leaf problems, so here is a list of the four most fundamental skills in this course:

%\begin{enumerate}
%   \item Check your answers by plugging in numbers. Specifically, be able to check if a given number is a solution of an equation. Also, be able to check if two expressions are equivalent. By hand if you can and by calculator if not.
%   \item Solve any kind of equation with one variable, by graphing, on your graphing calculator.
%   \item Use the given graph of a function or equation (which you can graph on your calculator) to answer basic questions about the function/equation. (Examples of basic questions: Find $f(3)$. Find $x$ so that $y=0$. What is the maximum value of $y$, etc.)
%   \item Do basic word problems. (Examples: Given description of $x,y$ relationship, write equation for $y$ in terms of $x$, and find value of $x$ or $y$  given the other. Or, given equation, and given $x$, find $y$ and interpret the result. Interpret specific $(x,y)$ points, given a context.)
%\end{enumerate}

%Right now, you probably don't understand some of this, but we'll learn what it all means. In the \texttt{hw\_items.pdf} file I'll try to label problems as FS1, FS2, FS3, or FS4, so you know it's an important problem.

%\newpage

   

\end{document}


   
   

